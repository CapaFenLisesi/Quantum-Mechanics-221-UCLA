\documentclass[12pt]{article}
\usepackage{amsfonts, amsmath, amsthm}

% a matter of taste
\setlength{\parskip}{1ex}
\setlength{\parindent}{0pt}

\newtheorem*{exer}{Exercise}

% A few simple macros for group theory.
\newcommand{\floor}{\text{floor }}
\newcommand{\img}{\text{img }}
\newcommand{\lcm}{\text{lcm }}
\newcommand{\aut}{\text{Aut }}
\newcommand{\cycle}[1]{(\mathbf{#1})}

\begin{document}

\textbf{Homework number -- class name here} \\

\hrule

% Problem list
\begin{minipage}{.80\linewidth}
    \flushleft
    Ch 11: 1.1, 1.2, 1.3, 1.8, 2.1, 3.1, 3.2, 3.3(a, d), 3.5, 3.6, 3.8,
    3.12, 3.13 \\
    % ``Pre-lecture problems''
    Pre-lect:  \\
\end{minipage}
\begin{minipage}{.20\linewidth}
    \flushright
    % whoami
    Blake Griffith
\end{minipage}

% % % % % % % % % % % % % % % % % % % % % % % % % % % % % % % % % % % % 
% Problems
% % % % % % % % % % % % % % % % % % % % % % % % % % % % % % % % % % % % 

\begin{exer}[11.1.1]

    Prove that $7 + 2^{1/3}$ and $\sqrt{3} + \sqrt{-5}$ are algebraic
    numbers.

\end{exer}

\begin{proof}

    Recall that a number $x$ is \textit{algebraic} if it is the solution
    to the equation $0 = a_n x^n + \dots + a_1 x + a_0$ for any set of
    $a$'s in $\mathbb{Z}$.

    So we must construct a polynomial for which the given numbers are
    roots. 

    For if we consider $7 + 2^{1/3}$ as the expression $x + y^{1/3}$ we
    would want to construct a polynomial which has no terms with the
    non-integer powers of $y$ eliminated which yields an integer we can
    then choose $a_0$ to be minus this integer.

    Consider 
    \[
        (x + y^{1/3})^3 = x^3 + 3x^2y^{1/3} + 3xy^{2/3} + y
    \]
    So we can choose the $a_2$ term to be $-3x = -21$ giving
    \[
        (x + y^{1/3})^3 - 3x(x + y^{1/3})^2 = -2x^3 - 3x^2y^{1/3} + y
    \]
    So no we can choose the $a_1$ term to be $3x^2 = 147$ giving
    \[
        (x + y^{1/3})^3 - 3x(x + y^{1/3})^2 + 3x^2(x + y^{1/3}) = x^3 +
        y
    \]
    We then choose $a_0$ to be $x^3 + y = 345$. So $7 + 2^{1/3}$ is
    algebraic because it is the root of the polynomial $y = x^3 - 21x^2
    + 147x + 345$.

    For the next part let $x = \sqrt{3} + \sqrt{-5}$ then:
    \[
        x = \sqrt{3} + \sqrt{-5} \implies x^2 = -2 + 2\sqrt{-15}
        \implies (x^2 + 2)^2 = (2\sqrt{15})^2 \implies x^4 + 4x^2 + 64 =
        0
    \]
    So $\sqrt{3} + \sqrt{-5}$ is algebraic because it is the root of the
    equation $y = x^4 + 4x^2 + 64$.

\end{proof}

% % % % % % % % % % % % % % % % % % % % % % % % % % % % % % % % % % % % 

\begin{exer}[11.1.8]

    Determine the units in

    \begin{enumerate}
        \item $\mathbb{Z}/12\mathbb{Z}$
        \item $\mathbb{Z}/8\mathbb{Z}$
        \item $\mathbb{Z}/n\mathbb{Z}$
    \end{enumerate}

\end{exer}

\begin{proof}

    Recall a \textit{unit} is a element of a ring that has a
    multiplicative inverse.

    \begin{enumerate}
        \item The set of units in $\mathbb{Z}/12\mathbb{Z}$ is $\{1, 2,
            3, 4, 6, 8, 9, 10\}$.

        \item The set of units in $\mathbb{Z}/8\mathbb{Z}$ is $\{1, 2,
            4, 6\}$.

        \item The pattern in the previous two problems indicates that
            elements which are coprime to the order of the quotient
            group are \textit{not} units. This is reasonable, if we
            consider a number $q$ that does not divide $n$. Then the
            smallest number multiplied with $p$ that is congruent to $1$
            is LCM($q,n$). But since $q$ does not divide $n$ this is
            $qn$.  However there are only elements in the ring less than
            $n$, so $q$ cannot be a unit.

            So the units are $\mathbb{Z}/n\mathbb{Z} - \phi(n)$. Where
            $\phi$ is Euler's totient function.

    \end{enumerate}
\end{proof}

\end{document}
