% ***********************************************************
% ******************* PHYSICS HEADER ************************
% ***********************************************************
% Version 2
\documentclass[11pt]{article}
\usepackage{amsmath} % AMS Math Package
\usepackage{amsthm} % Theorem Formatting
\usepackage{amssymb}	% Math symbols such as \mathbb
\usepackage{graphicx} % Allows for eps images
\usepackage{multicol} % Allows for multiple columns
\usepackage[dvips,letterpaper,margin=0.75in,bottom=0.5in]{geometry}
 % Sets margins and page size
\pagestyle{empty} % Removes page numbers
\makeatletter % Need for anything that contains an @ command
\renewcommand{\maketitle} % Redefine maketitle to conserve space
{ \begingroup \vskip 10pt \begin{center} \large {\bf \@title}
	\vskip 10pt \large \@author \hskip 20pt \@date \end{center}
  \vskip 10pt \endgroup \setcounter{footnote}{0} }
\makeatother % End of region containing @ commands
\renewcommand{\labelenumi}{(\alph{enumi})} % Use letters for enumerate
% \DeclareMathOperator{\Sample}{Sample}
\let\vaccent=\v % rename builtin command \v{} to \vaccent{}
\renewcommand{\v}[1]{\ensuremath{\mathbf{#1}}} % for vectors
\newcommand{\gv}[1]{\ensuremath{\mbox{\boldmath$ #1 $}}}
% for vectors of Greek letters
\newcommand{\uv}[1]{\ensuremath{\mathbf{\hat{#1}}}} % for unit vector
\newcommand{\abs}[1]{\left| #1 \right|} % for absolute value
\newcommand{\avg}[1]{\left< #1 \right>} % for average
\let\underdot=\d % rename builtin command \d{} to \underdot{}
\renewcommand{\d}[2]{\frac{d #1}{d #2}} % for derivatives
\newcommand{\dd}[2]{\frac{d^2 #1}{d #2^2}} % for double derivatives
\newcommand{\pd}[2]{\frac{\partial #1}{\partial #2}}
% for partial derivatives
\newcommand{\pdd}[2]{\frac{\partial^2 #1}{\partial #2^2}}
% for double partial derivatives
\newcommand{\pdc}[3]{\left( \frac{\partial #1}{\partial #2}
 \right)_{#3}} % for thermodynamic partial derivatives
\newcommand{\ket}[1]{\left| #1 \right>} % for Dirac bras
\newcommand{\bra}[1]{\left< #1 \right|} % for Dirac kets
\newcommand{\braket}[2]{\left< #1 \vphantom{#2} \right|
 \left. #2 \vphantom{#1} \right>} % for Dirac brackets
\newcommand{\matrixel}[3]{\left< #1 \vphantom{#2#3} \right|
 #2 \left| #3 \vphantom{#1#2} \right>} % for Dirac matrix elements
\newcommand{\grad}[1]{\gv{\nabla} #1} % for gradient
\let\divsymb=\div % rename builtin command \div to \divsymb
\renewcommand{\div}[1]{\gv{\nabla} \cdot #1} % for divergence
\newcommand{\curl}[1]{\gv{\nabla} \times #1} % for curl
\let\baraccent=\= % rename builtin command \= to \baraccent
\renewcommand{\=}[1]{\stackrel{#1}{=}} % for putting numbers above =
\newtheorem{prop}{Proposition}
\newtheorem{thm}{Theorem}[section]
\newtheorem{lem}[thm]{Lemma}
\theoremstyle{definition}
\newtheorem{dfn}{Definition}
\theoremstyle{remark}
\newtheorem*{rmk}{Remark}

% ***********************************************************
% The following added by Blake Griffith
% ***********************************************************

% shorthand for a 2 by 1 matrix
\newcommand{\twobyone}[2] {
    \begin{pmatrix}
        #1 \\ #2
    \end{pmatrix}
}
% ***********************************************************
% ********************** END HEADER *************************
% ***********************************************************

\usepackage{amsfonts, amsmath, amsthm}

% a matter of taste
\setlength{\parskip}{1ex}
\setlength{\parindent}{0pt}

\newtheorem*{exer}{Exercise}

% A few simple macros for group theory.
\newcommand{\floor}{\text{floor }}
\newcommand{\img}{\text{img }}
\newcommand{\lcm}{\text{lcm }}
\newcommand{\aut}{\text{Aut }}
\newcommand{\cycle}[1]{(\mathbf{#1})}

\begin{document}

\textbf{Homework 6 -- Quantum Mechanics} \\

\hrule

% Problem list
\begin{minipage}{.80\linewidth}
    \flushleft
    Ch 3: 1, 2, 4, 16 \\
\end{minipage}
\begin{minipage}{.20\linewidth}
    \flushright
    % whoami
    Blake Griffith
\end{minipage}

% % % % % % % % % % % % % % % % % % % % % % % % % % % % % % % % % % % %
% Problems
% % % % % % % % % % % % % % % % % % % % % % % % % % % % % % % % % % % %

\begin{exer}[3.1]

    Find the eigenvalues and eigenvectors of
    $\sigma_y = \begin{pmatrix}
        0 & -i \\
        i & 0
    \end{pmatrix}$
    . Suppose an electron is in the spin state $
    \twobyone{\alpha}{\beta} $. If $S_y$ is measured, what is the
    probability of the result $\hbar / 2$?


\end{exer}

\begin{proof}

    \begin{enumerate}
        \item We begin by solving the charachteristic equation to find
            the eigenvalues of $\sigma_y$:
            \[
            \begin{vmatrix}
                -\lambda & -i \\
                i & -\lambda
            \end{vmatrix} = \lambda^2 -  1 = 0 \implies
            \boxed{\lambda = \pm 1}
            \]
            Now we find the corresponding eigenvectors, for $\lambda =
            -1$:
            \[
            \sigma_y \twobyone{a}{b} = \twobyone{-a}{-b} \implies
            -ib = -a \qquad ai = -b \implies
            \boxed{\v{X}(\lambda = -1) =
            \frac{1}{\sqrt{2}}\twobyone{1}{-i}}
            \]
            for $\lambda = 1$:
            \[
            \sigma_y \twobyone{a}{b} = \twobyone{a}{b} \implies
            -ib = a \qquad ai = b \implies
            \boxed{\v{X}(\lambda = 1) =
            \frac{1}{\sqrt{2}}\twobyone{1}{i}}
            \]

        \item
            The probability of measuring $S_y = \hbar/2$ (spin up in the
            $y$ basis) is $\left|\bra{\psi_{y+}} S_y
            \ket{\psi}\right|^2$. Where is $\bra{\psi_{y+}}$ is the
            eigenstate corresponding to the
            $S_y$ ``up'' eigenvalue ($\lambda = 1$). Which we found
            above since $S_y = \frac{\hbar}{2}\sigma_y$. So we have
            \[
                \left| \bra{\psi_{y+}} S_y \ket{\psi} \right|^2 \implies
                \frac{\hbar^2}{8} \left|
                    \begin{pmatrix} 1 & i \end{pmatrix}
                    \begin{pmatrix} 0 & -i \\ i & 0 \end{pmatrix}
                    \twobyone{\alpha}{\beta} \right|^2 \implies
                    \boxed{\frac{\hbar^2}{8}(\alpha^2 + \beta^2)}
            \]

    \end{enumerate}


\end{proof}

% % % % % % % % % % % % % % % % % % % % % % % % % % % % % % % % % % % %

\begin{exer}[3.2]

    Find by explicit construction using Pauli matrices, the eigenvalues
    for the Hamiltonian
    \[
        H = -\frac{2\mu}{\hbar} \v{S} \cdot \v{B}
    \]
    for a spin $\frac{1}{2}$ particle in the presence of a magnetic
    field $\v{B} = B_x \hat{\v{x}} + B_y \hat{\v{y}} + B_z \hat{\v{z}}$.

\end{exer}

\begin{proof}

    We rotate our coordinate system so that $\v{B}$ points in the
    $\hat{\v{z}}$ direction at the point of the particle. Then our
    Hamiltonian becomes
    \[
        H = - \frac{2 \mu B}{\hbar} S_z = \omega S_z
    \]
    Recall the eigenvalue for $S_z$ is $\pm \frac{\hbar}{2}$, so at $t =
    0$ we have:
    \[
        H\ket{\psi} = \mp \frac{2 \mu B}{2}\ket{\psi}
    \]
    To calculate the time evolution note that the time evolution
    operator is:
    \[
        U(t) = \exp{\frac{-i H t}{\hbar}} = \exp{\frac{-i \omega t
        S_z}{\hbar}} \implies U(t)\ket{\psi} = \exp{\frac{\mp i \omega
        t}{2}}
    \]
    Combining these:
    \[
        H\ket{\psi(t)} = H U(t) \ket{\psi} = \boxed{\mp \frac{2 \mu
        B}{2} \exp \left(\frac{\mp i \omega t}{2}\right)\ket{\psi}}
    \]
    Where do Pauli matrices come into this?

\end{proof}

% % % % % % % % % % % % % % % % % % % % % % % % % % % % % % % % % % % %

\begin{exer}[3.4]

    The spin-dependent Hamiltonian of an electron-positron system in the
    presence of a uniform magnetic field in the $z$-direction can be
    written as

    \[
        H = A \v{S}^{e^-} \cdot \v{S^{e^+}} + \frac{e B}{mc}
        \left(S^{e^-}_{z} - S^{e^+}_{z} \right)
    \]

    Suppose the spin function of the system is given by $\chi^{e^-}_+
    \chi^{e^+}_-$.
    \begin{enumerate}
        \item Is this an eigenfunction of $H$ in the limit $A \to
            0$, $eB/mc \not= 0$? If it is, what is the energy eigenvalue?
            If it is not, what is the expectation value of $H$?
        \item Solve the same problem when $eB/mc \to 0$, $A \not= 0$.
    \end{enumerate}

\end{exer}

\begin{proof}

    \begin{enumerate}
        \item With $A \to 0$ the hamiltonian is just the term from the
            electron and positron independent of interactions:
            \[
                H = \frac{e B}{mc} \left(S^{e^-}_{z} - S^{e^+}_{z}
                \right)
            \]

        \item With $eB/mc \to 0$, $A \not= 0$ we have:
            \[
                H = A \v{S}^{e^-} \cdot \v{S^{e^+}}
            \]
    \end{enumerate}

\end{proof}

% % % % % % % % % % % % % % % % % % % % % % % % % % % % % % % % % % % %

\begin{exer}[3.16]

    Show that the orbital angular-momentum operator $\v{L}$ commutes
    with both the operators $\v{p}^2$ and $\v{x}^2$; that is, prove
    (3.7.2).

\end{exer}

\begin{proof}

    \begin{enumerate}
        \item Looking at $[\v{L}, \v{p}^2] = \v{L}$ we see that each
                $p_i$ much commute with each $\v{L}$ component so we
                have
            \[
                [L_x, p^2_x + p^2_y + p^2_z] = [(\v{x}\times\v{p})_x, +
                p^2_y + p^2_z] = [x_y p_z - x_z p_y,  p^2_x + p^2_y +
                p^2_z]
            \]
            So we expressions of the form $[x_i p_j, p^2_k] =
            \delta_{ij} i \hbar (2 p_k p_j)$, we prove this:
            \[
                [x_i p_j, p^2_k] = [x_i, p^2_k]p_j + x_i [p_k, p^2_j]
            \]
            the second term is zero so we have:
            \[
                [x_i p_j, p^2_k] = [x_i, p^2_k]p_j = [x_i, p_k p_k]p_j =
                [x_i, p_k]p_k p_j + p_k[ x_i, p_k] p_j = 2 i \hbar p_k
                p_j \delta_{ik}
            \]
            So for each angular momentum component we have:
            \[
                [L_x, p^2_x + p^2_y + p^2_z] = [x_y p_z - x_z p_y,
                p^2_x + p^2_y + p^2_z] = [x_y p_z, p^2_y] - [x_z p_y,
                p^2_z] = i2\hbar (p_y p_z  - p_z p_y) = i2\hbar [p_y,
                p_z] = 0
            \]
            Which implies $[\v{L}, \v{p}^2] = 0$.

        \item $[ \v{L}, \v{x}^2]$ reduces to a similar relation:
            \[
                [L_z, \v{x}^2] = [xp_y - yp_x, x^2 + y^2 + z^2]
            \]
            gives us expressions like $[x_i p_j, x^2_{k}] = x_i x_k (-i
            2 \hbar \delta_{jk} )$, proof:
            \[
                [x_i p_j, x^2_k] = [x_i, x^2_k]p_j + x_i [p_j, x^2_k]
            \]
            The first term is zero so:
            \[
                x_i [p_j, x^2_k] = x_i ([p_j, x_k] x_k + x_k [p_j, x_k]) =
                x_i x_k (- i 2 \hbar \delta_{jk})
            \]
            Applying this identity we get
            \[
                [L_z, \v{x}^2] = [xp_y, y^2] - [y p_x, x^2] =
                -2i\hbar(xy- yx) = -2i\hbar [x, y] = 0
            \]
            Which implies $[\v{L}, \v{x}^2]$.

    \end{enumerate}

\end{proof}

\end{document}
