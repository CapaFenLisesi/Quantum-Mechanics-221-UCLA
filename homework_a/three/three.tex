% ***********************************************************
% ******************* PHYSICS HEADER ************************
% ***********************************************************
% Version 2
\documentclass[11pt]{article}
\usepackage{amsmath} % AMS Math Package
\usepackage{amsthm} % Theorem Formatting
\usepackage{amssymb}	% Math symbols such as \mathbb
\usepackage{graphicx} % Allows for eps images
\usepackage{multicol} % Allows for multiple columns
\usepackage[dvips,letterpaper,margin=0.75in,bottom=0.5in]{geometry}
 % Sets margins and page size
\pagestyle{empty} % Removes page numbers
\makeatletter % Need for anything that contains an @ command
\renewcommand{\maketitle} % Redefine maketitle to conserve space
{ \begingroup \vskip 10pt \begin{center} \large {\bf \@title}
	\vskip 10pt \large \@author \hskip 20pt \@date \end{center}
  \vskip 10pt \endgroup \setcounter{footnote}{0} }
\makeatother % End of region containing @ commands
\renewcommand{\labelenumi}{(\alph{enumi})} % Use letters for enumerate
% \DeclareMathOperator{\Sample}{Sample}
\let\vaccent=\v % rename builtin command \v{} to \vaccent{}
\renewcommand{\v}[1]{\ensuremath{\mathbf{#1}}} % for vectors
\newcommand{\gv}[1]{\ensuremath{\mbox{\boldmath$ #1 $}}}
% for vectors of Greek letters
\newcommand{\uv}[1]{\ensuremath{\mathbf{\hat{#1}}}} % for unit vector
\newcommand{\abs}[1]{\left| #1 \right|} % for absolute value
\newcommand{\avg}[1]{\left< #1 \right>} % for average
\let\underdot=\d % rename builtin command \d{} to \underdot{}
\renewcommand{\d}[2]{\frac{d #1}{d #2}} % for derivatives
\newcommand{\dd}[2]{\frac{d^2 #1}{d #2^2}} % for double derivatives
\newcommand{\pd}[2]{\frac{\partial #1}{\partial #2}}
% for partial derivatives
\newcommand{\pdd}[2]{\frac{\partial^2 #1}{\partial #2^2}}
% for double partial derivatives
\newcommand{\pdc}[3]{\left( \frac{\partial #1}{\partial #2}
 \right)_{#3}} % for thermodynamic partial derivatives
\newcommand{\ket}[1]{\left| #1 \right>} % for Dirac bras
\newcommand{\bra}[1]{\left< #1 \right|} % for Dirac kets
\newcommand{\braket}[2]{\left< #1 \vphantom{#2} \right|
 \left. #2 \vphantom{#1} \right>} % for Dirac brackets
\newcommand{\matrixel}[3]{\left< #1 \vphantom{#2#3} \right|
 #2 \left| #3 \vphantom{#1#2} \right>} % for Dirac matrix elements
\newcommand{\grad}[1]{\gv{\nabla} #1} % for gradient
\let\divsymb=\div % rename builtin command \div to \divsymb
\renewcommand{\div}[1]{\gv{\nabla} \cdot #1} % for divergence
\newcommand{\curl}[1]{\gv{\nabla} \times #1} % for curl
\let\baraccent=\= % rename builtin command \= to \baraccent
\renewcommand{\=}[1]{\stackrel{#1}{=}} % for putting numbers above =
\newtheorem{prop}{Proposition}
\newtheorem{thm}{Theorem}[section]
\newtheorem{lem}[thm]{Lemma}
\theoremstyle{definition}
\newtheorem{dfn}{Definition}
\theoremstyle{remark}
\newtheorem*{rmk}{Remark}

% ***********************************************************
% The following added by Blake Griffith
% ***********************************************************

% shorthand for a 2 by 1 matrix
\newcommand{\twobyone}[2] {
    \begin{pmatrix}
        #1 \\ #2
    \end{pmatrix}
}
% ***********************************************************
% ********************** END HEADER *************************
% ***********************************************************

\usepackage{amsfonts, amsmath, amsthm}

% a matter of taste
\setlength{\parskip}{1ex}
\setlength{\parindent}{0pt}

\newtheorem*{exer}{Exercise}

% A few macros: get better qm physics ones:
\newcommand{\lam}{\lambda}

\begin{document}

\textbf{Homework 3 -- Quantum Mechanics} \\

\hrule

% Problem list
\begin{minipage}{.80\linewidth}
    \flushleft
    Ch 2: 17, 18, 19, 20, 22 \\
\end{minipage}
\begin{minipage}{.20\linewidth}
    \flushright
    % whoami
    Blake Griffith
\end{minipage}

% % % % % % % % % % % % % % % % % % % % % % % % % % % % % % % % % % % % 
% Problems
% % % % % % % % % % % % % % % % % % % % % % % % % % % % % % % % % % % % 

\begin{exer}[2.17]

    Consider again a one-dimensional simple harmonic oscillator. Do the
    following algebraically, that is, without using wave functions.

    \begin{enumerate}
        \item Construct a linear combination of $\ket{0}$ and $\ket{1}$
            such that $\avg{x}$ is as large as possible.

        \item Suppose the oscillator is in the state constructed in (a)
            at $t=0$. What is the state vector for $t>0$ in the
            Schr\"{o}dinger picture? Evaluate the expectation value
            $\avg{x}$ as a function of for $t>0$ using (i) the
            Schr\"{o}dinger picture and (ii) the Heisenberg picture.

        \item Evaluate $\avg{(\Delta x)^2}$ as a function of time using
            either picture.

    \end{enumerate}

\end{exer}

\begin{proof}

    Put the proof here.

\end{proof}

% % % % % % % % % % % % % % % % % % % % % % % % % % % % % % % % % % % % 

\begin{exer}[2.18]

    Show for the one-dimensional simple harmonic oscillator
    \[
        \bra{0} e^{ikx} \ket{0} = \exp[- k^2 \bra{0} x^2 \ket{0} / 2]
    \]
    where $x$ is the positoin \it{operator}
    
\end{exer}

\begin{proof}

    shits

\end{proof}

% % % % % % % % % % % % % % % % % % % % % % % % % % % % % % % % % % % % 

\begin{exer}[2.19]

    A coherent state of a one-dimensional simple harmonic oscillator is
    defined to be an eigen state of the (non-Hermition) annihilation
    operator $a$:
    \[
        a\ket{\lam} = \lam \ket{\lam}
    \]
    
    where $\lam$ is, in general, a complex number.
    \begin{enumerate}
        \item Prove that 
            \[
                \ket{\lam} = e^{\abs{\lam}^2 / 2} e^{\lam
                a^\dagger} \ket{0}
            \]

        \item Prove the minimum uncertainty relation for such a state.

        \item Write $\ket{\lam}$ as 
            \[
                \ket{\lam} = \sum^{\infty}_{n = 0} f(n) \ket{n}
            \]

            Show that the distribution $\abs{f(n)}^2$ with respect to
            $n$ is of the Poisson form. Find the most probabble value of
            $n$, hence of $E$.

        \item Sshow that a coherent state can also be obtained by
            applying the translation (finite-displacement) operator
            $e^-ipl/\hbar$ (where $p$ is the momentum operator, and
            $l$ is the displacement distance) to the ground state. (See
            also Gottfried 1966, 262-64.)

    \end{enumerate}
\end{exer}

\begin{proof}

    shits

\end{proof}

% % % % % % % % % % % % % % % % % % % % % % % % % % % % % % % % % % % % 

\begin{exer}[2.20]

    Let
    \[
        J_{\pm} = \hbar a^{\dagger}_{\pm} a_{\mp}\text{,} \qquad
        J_z = \frac{\hbar}{2}\left(a^{\dagger}_{+} - a^{\dagger}_{-}
        a_{-}\right)\text{,} \qquad
        N = a^{\dagger}_{+}a_+ + a^{\dagger}_{-} a_-
    \]
    
    where $a_{\pm}$ and $a^{\dagger}_{\pm}$ are the annihilation and
    creation operators of two \it{independent} simple harmonic
    oscillator satisfying the usal simple hormonic oscillator
    commutation relations. Prove
    \[
        [J_z, J_{\pm}] = \pm \hbar J_\pm , \qquad
        [\v{J}^2, J_z] = 0, \qquad
        \v{J}^2 = \left(\frac{\hbar^2}{2}\right) N \left[ \frac{N}{2} +
        1 \right]
    \]

\end{exer}

\begin{proof}

    shits

\end{proof}

% % % % % % % % % % % % % % % % % % % % % % % % % % % % % % % % % % % % 

\begin{exer}[2.22]

    Consider a particle of mass $m$ subject to a one-dimensional
    potential of the following form:
    \[
        V =
        \begin{cases}
            \frac{1}{2} k x^2 & \text{for } x > 0\\
            \infty & \text{for } x < 0
        \end{cases}
    \]
    \begin{enumerate}
        \item What is the ground-state energy?
        \item What is the expectation value $\avg{x^2}$ for the ground
            state?
    \end{enumerate}

\end{exer}

\begin{proof}

    shits

\end{proof}

\end{document}
